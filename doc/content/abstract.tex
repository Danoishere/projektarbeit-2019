%%%%%%%%%%%%%%%%%%%%%%%%%%%%%%%%%%%%%%%%%%%%%%%%%%%%%%%%%%%%%%%%%
%  _____   ____  _____                                          %
% |_   _| /  __||  __ \    Institute of Computitional Physics   %
%   | |  |  /   | |__) |   Zuercher Hochschule Winterthur       %
%   | |  | (    |  ___/    (University of Applied Sciences)     %
%  _| |_ |  \__ | |        8401 Winterthur, Switzerland         %
% |_____| \____||_|                                             %
%%%%%%%%%%%%%%%%%%%%%%%%%%%%%%%%%%%%%%%%%%%%%%%%%%%%%%%%%%%%%%%%%
%
% Project     : LaTeX doc Vorlage für Windows ProTeXt mit TexMakerX
% Title       : 
% File        : abstract.tex Rev. 00
% Date        : 23.4.12
% Author      : Remo Ritzmann
% Feedback bitte an Email: remo.ritzmann@pfunzle.ch
%
%%%%%%%%%%%%%%%%%%%%%%%%%%%%%%%%%%%%%%%%%%%%%%%%%%%%%%%%%%%%%%%%%

\thispagestyle{empty}

\chapter*{Zusammenfassung}\label{chap.zusammenfassung}
\begin{otherlanguage}{nswissgerman}
%1. Einleitung («Problemstellung»): Definiert die Problematik und begründet die Relevanz der Arbeit. Die Situation (eine problematische Situation oder technische Problematik) und die wissenschaftliche/fachliche Untersuchungs-Fragestellung (folgt logischer- weise aus der festgestellten Problematik) werden kurz beschrieben.
Die steigende Anzahl der Pendler bring die Kapazität des Schienennetzes der Schweizerischen Bundesbahnen SBB immer mehr an seine Grenzen. Da der Ausbau der Infrastruktur nicht mit den Passagierzahlen mithalten kann, bemüht sich die SBB, mehr Züge in dichteren Abständen auf ihr Netz zu bringen.
Für die Planung des dichteren Zugverkehrs experimentiert die SBB auch mit Technologien des maschinellen Lernens, insbesondere mit Reinforcement Learning (RL). Dabei soll ein Algorithmus selbstständig Strategien erlernen, um das gegebene Problem zu lösen. In diesem Fall handelt es sich dabei um das Steuern von Zügen, welche möglichst zeitnah zu ihrem Ziel geführt werden sollen. Kollisionen gilt es dabei zu verhindern und potenzielle Hindernisse wie defekte Züge müssen umfahren werden. Auch können Züge mit verschiedenen Geschwindigkeiten unterwegs sein, was bei der Planung berücksichtigt werden soll.\\
Diese Arbeit ist dabei ein Beitrag zur sogenannten Flatland Challenge, einem Wettbewerb der SBB und der Crowdsourcing-Plattform AICrowd, bei welchem das beschriebene Problem mittels einer zur Verfügung gestellten Simulation des Schienennetzes gelöst werden soll. Der Wettbewerb ist aufgeteilt in zwei Runden mit wachsendem Schwierigkeitslevel.
Diese Schienensimulation erlaubt es, eigene Lösungen für das Problem zu trainieren und zu evaluieren. Die Schwierigkeit dabei ist, die Züge so zu steuern, dass sie sich auch in komplexen Situationen nicht blockieren. Dazu ist primär die Zusammenarbeit der Züge von grosser Bedeutung.\\
%3. Vorgehen («Problembehandlung»): Informiert über das Vorgehen und über die Untersuchungsanlage.
Der präsentierte Lösungsansatz verwendet den Asynchronous Advantage Actor Critic Algorithmus (A3C), einen der derzeit besten RL-Algorithmen, welcher verteiltes Lernen ermöglicht. Der präsentierte Ansatz orientiert sich dabei stark an der Ausgangslösung von Stephan Huschauer.
Nach der ersten lauffähigen Version wurde der Ansatz nach und nach mit unterschiedlichen Features erweitert, wie beispielsweise einem veränderten Kontrollmechanismus der Züge, Curriculum Learning oder verteiltem Lernen über mehrere Rechner.
Um den Fortschritt zu quantifizieren wurden für sämtliche Änderungen Experimente durchgeführt. Durch das Hinzufügen unterschiedlicher Features konnte eine signifikante Verbesserung der Performanz verzeichnet werden. In Runde 1 war es dabei möglich, die Ankunftsrate von den 16.6\% der Ausgangslösung auf 48.9\% zu verbessern.
Für die anspruchsvolleren Eisenbahnnetze der 2. Runde, für welche keine Ausgangslösung existiert, konnte dank weiteren Verbesserungen ebenfalls ein respektables Resultat erzielt werden. Die präsentierte Lösung konnte eine Ankunftsrate von 29.1\% erreichen, was zum Evaluierungszeitpunkt dem 4. Platz von insgesamt 24 teilnehmenden Teams entsprach. Für einen praktischen Einsatz ist die präsentierte Lösung jedoch noch nicht geeignet, dies bedarf weiterer Forschung.
\end{otherlanguage}

\NewPage
\thispagestyle{empty}
\chapter*{Abstract}\label{abstract}
The increasing number of commuters is pushing the capacity of the Swiss Federal Railways (SBB) rail network to its limits. Since the expansion of the infrastructure cannot keep up with the increasing number of passengers, SBB is exploring new ways to bring more trains onto its network. To achieve this goal, SBB is also experimenting with machine learning technologies such as reinforcement learning (RL). 
The goal for this type of algorithm is to independently learn strategies  in order to solve the problem at hand, in our case the challenge of guiding trains to their assigned destinations. Besides not colliding with other trains, it is also necessary to avoid potential obstacles such as defective trains and to take different speed profiles into account.\\
This work is a contribution the Flatland Challenge, a competition published by SBB and the crowdsourcing platform AICrowd, which aims to solve the problem described by using a provided simulation of the rail network. The competition is divided into two rounds with increasing difficulty. This rail simulation allows to train and evaluate own solutions for the given problem. The main task is to control the trains in cooperative way so they do not block each other even in complex situations. The selected solution uses the Asynchronous Advantage Actor Critic Algorithm (A3C), a state of the art RL algorithm which supports distributed learning. The approach presented is based on the baseline solution presented by Stephan Huschauer. After establishing the first running version, the approach was improved with an array of new features such as a modified train control mechanism, curriculum learning or distributed learning over multiple computers. To quantify the progress, experiments were conducted for all applied changes. By adding these features, a significant improvement could be achieved. In round one it was possible to improve the train arrival rate from 16.6\% of the baseline solution to 48.9\%.
For the more demanding second round, there is no baseline available. The solution presented was able to achieve a train arrival rate of 29.1\%, which corresponded at submission time to the 4th place of a total of 24 participating teams.
However, the solution presented is not yet suitable for practical use and requires further reseach to achieve a useful performance in a real world scenario.

\chapter*{Preface}\label{preface}
We would like to give special thanks to:
\begin{itemize}
    \item Andreas Weiler and Thilo Stadelmann for their great support during this work, for their helpful tips and for pushing us into the right direction.\\
    We are grateful for the opportunity to dive deep into the field of reinforcement learning as part of this project.
    \item Remo Maurer for being very generous with providing computing infrastructure, especially the infrastructure test server.
\end{itemize}

