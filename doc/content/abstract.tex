%%%%%%%%%%%%%%%%%%%%%%%%%%%%%%%%%%%%%%%%%%%%%%%%%%%%%%%%%%%%%%%%%
%  _____   ____  _____                                          %
% |_   _| /  __||  __ \    Institute of Computitional Physics   %
%   | |  |  /   | |__) |   Zuercher Hochschule Winterthur       %
%   | |  | (    |  ___/    (University of Applied Sciences)     %
%  _| |_ |  \__ | |        8401 Winterthur, Switzerland         %
% |_____| \____||_|                                             %
%%%%%%%%%%%%%%%%%%%%%%%%%%%%%%%%%%%%%%%%%%%%%%%%%%%%%%%%%%%%%%%%%
%
% Project     : LaTeX doc Vorlage für Windows ProTeXt mit TexMakerX
% Title       : 
% File        : abstract.tex Rev. 00
% Date        : 23.4.12
% Author      : Remo Ritzmann
% Feedback bitte an Email: remo.ritzmann@pfunzle.ch
%
%%%%%%%%%%%%%%%%%%%%%%%%%%%%%%%%%%%%%%%%%%%%%%%%%%%%%%%%%%%%%%%%%

\thispagestyle{empty}
\chapter*{Zusammenfassung}\label{chap.zusammenfassung}
Zusammenfassung in Deutsch

Diese Arbeit befasst sich mit der Steuerung von komplexen Zugverkehrssystemen mittels Reinforcement Learning.
Die Aufgabenstellung hat die Schweizerische Bundesbahn im Rahmen einer Challenge auf AICrowd ausgeschrieben.
Im Fokus steht eine Lösung zu entwicklen, welche die Verspätung bei technischen Störrungen oder defekten Zügen verringert und es ihnen zudem ermöglicht in Zukunft mehr Züge auf die gleiche Infrastruktur zu bringen.


%1. Einleitung («Problemstellung»): Definiert die Problematik und begründet die Relevanz der Arbeit. Die Situation (eine problematische Situation oder technische Problematik) und die wissenschaftliche/fachliche Untersuchungs-Fragestellung (folgt logischer- weise aus der festgestellten Problematik) werden kurz beschrieben.

%2. Methodische Einordnung der Arbeit: Art, Datenbasis und Ziel der Arbeit. Die Untersu- chungsmethode (Umfrage, Analyse, Versuch, Test etc.), das untersuchte „Material“ und das dabei verfolgte Ziel wird thematisiert.

%3. Vorgehen («Problembehandlung»): Informiert über das Vorgehen und über die Unter- suchungsanlage.

%4. Ergebnis («Problemlösung»): Beschreibt die wichtigsten Resultate, Erkenntnisse, of- fenen Fragen. Die Ergebnisse werden aufgeführt und dabei wird auf die unter 1. auf- geführte Fragestellung Bezug genommen (konnte sie beantwortet werden, gibt es of- fene Punkte?).


\newpage
\thispagestyle{empty}
\chapter*{Abstract}\label{abstract}
Abstract in English



\chapter*{(Deutschsprachiges Management Summary)}\label{ManagementSummaryDE}



\chapter*{(Englischsprachiges Management Summary)}\label{ManagementSummaryEN}



\chapter*{Vorwort}\label{vorwort}
Stellt den persönlichen Bezug zur Arbeit dar und spricht Dank aus.