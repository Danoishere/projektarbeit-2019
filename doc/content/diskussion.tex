%%%%%%%%%%%%%%%%%%%%%%%%%%%%%%%%%%%%%%%%%%%%%%%%%%%%%%%%%%%%%%%%%
%  _____   ____  _____                                          %
% |_   _| /  __||  __ \    Institute of Computitional Physics   %
%   | |  |  /   | |__) |   Zuercher Hochschule Winterthur       %
%   | |  | (    |  ___/    (University of Applied Sciences)     %
%  _| |_ |  \__ | |        8401 Winterthur, Switzerland         %
% |_____| \____||_|                                             %
%%%%%%%%%%%%%%%%%%%%%%%%%%%%%%%%%%%%%%%%%%%%%%%%%%%%%%%%%%%%%%%%%
%
% Project     : LaTeX doc Vorlage für Windows ProTeXt mit TexMakerX
% Title       : 
% File        : diskussion.tex Rev. 00
% Date        : 7.5.12
% Author      : Remo Ritzmann
% Feedback bitte an Email: remo.ritzmann@pfunzle.ch
%
%%%%%%%%%%%%%%%%%%%%%%%%%%%%%%%%%%%%%%%%%%%%%%%%%%%%%%%%%%%%%%%%%

\chapter{Discussion and Outlook}
\label{chap.diskussion}
\chaptermark{Discussion}
\section{Practicability in a Real World Scenario}\label{discussion_real_world}
While we were able to heavily improve the performance of the presented solution compared to the given baseline, it is still nowhere near practical applicability. While the presented evaluation tasks probably do not represent the a real world density on a rail network, also with a lower volume of traffic, train traffic would require more robust solutions with the primary objective of finding a solution for every train to reach its destination instead of optimizing the performance of a single agent.
\section{Ideas for Future Research}\label{discussion_research}
We have various ideas for futher research to optimize the performance
A promising approach could 
