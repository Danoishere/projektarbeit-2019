%%%%%%%%%%%%%%%%%%%%%%%%%%%%%%%%%%%%%%%%%%%%%%%%%%%%%%%%%%%%%%%%%
%  _____   ____  _____                                          %
% |_   _| /  __||  __ \    Institute of Computitional Physics   %
%   | |  |  /   | |__) |   Zuercher Hochschule Winterthur       %
%   | |  | (    |  ___/    (University of Applied Sciences)     %
%  _| |_ |  \__ | |        8401 Winterthur, Switzerland         %
% |_____| \____||_|                                             %
%%%%%%%%%%%%%%%%%%%%%%%%%%%%%%%%%%%%%%%%%%%%%%%%%%%%%%%%%%%%%%%%%
%
% Project     : LaTeX doc Vorlage für Windows ProTeXt mit TexMakerX
% Title       : 
% File        : einleitung.tex Rev. 00
% Date        : 23.4.12
% Author      : Remo Ritzmann
% Feedback bitte an Email: remo.ritzmann@pfunzle.ch
%
%%%%%%%%%%%%%%%%%%%%%%%%%%%%%%%%%%%%%%%%%%%%%%%%%%%%%%%%%%%%%%%%%

\chapter{Einleitung}\label{chap.einleitung}


\section{Ausgangslage}\label{ausgangslage}

\begin{itemize}
\item Nennt bestehende Arbeiten/Literatur zum Thema -> Literaturrecherche
\item Stand der Technik: Bisherige Lösungen des Problems und deren Grenzen
\item (Nennt kurz den Industriepartner und/oder weitere Kooperationspartner und dessen/deren Interesse am Thema Fragestellung)
%The target of this work was to let trains find the optimal and fastest path to their target station. \\
The indirect industry partner during this work was the Swiss Federal Railways (SBB AG) which created the challange on AICrowd \cite{Test}.\\ 
The challenge consists of 2 parts. Part 1 was about avoiding conflicts with multiple trains on their given environment. \\
The aim of part 2 was to optimize train traffic which includes train with different speeds, broken trains and less switchover facilities.\\
%TODO: think about writing something about current state at SBB? -> Stand der Technik
\end{itemize}



\section{Zielsetzung / Aufgabenstellung / Anforderungen}\label{zielsetzung}

\begin{itemize}
\item Formuliert das Ziel der Arbeit
\item Verweist auf die offizielle Aufgabenstellung des/der Dozierenden im Anhang
\item (Pflichtenheft, Spezifikation)
\item (Spezifiziert die Anforderungen an das Resultat der Arbeit)
\item (Übersicht über die Arbeit: stellt die folgenden Teile der Arbeit kurz vor)
\item (Angaben zum Zielpublikum: nennt das für die Arbeit vorausgesetzte Wissen)
\item (Terminologie: Definiert die in der Arbeit verwendeten Begriffe)
\end{itemize}

