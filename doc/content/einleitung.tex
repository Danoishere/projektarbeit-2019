\chapter{Einleitung}\label{chap.einleitung}


\section{Baseline}\label{baseline}

\begin{itemize}
\item Nennt bestehende Arbeiten/Literatur zum Thema -> Literaturrecherche
\item Stand der Technik: Bisherige Lösungen des Problems und deren Grenzen
\item (Nennt kurz den Industriepartner und/oder weitere Kooperationspartner und dessen/deren Interesse am Thema Fragestellung)
\end{itemize}
The indirect industry partner during this work was the Swiss Federal Railways (SBB AG) which created the challange on AICrowd \cite{aicrowd}.\\
The challenge consists of 2 parts. Part 1 was about avoiding conflicts with multiple trains (agents) on their given environment. \\
The aim of part 2 was to optimize train traffic which includes trains with different speeds, broken trains and less switchover facilities.\\
We could use Stefan Husters work as a foundation to build our solution for the challenge.%TODO: add Stefan Husters Work



\section{Zielsetzung / Aufgabenstellung / Anforderungen}\label{zielsetzung}

\begin{itemize}
\item Formuliert das Ziel der Arbeit
\item Verweist auf die offizielle Aufgabenstellung des/der Dozierenden im Anhang
\item (Pflichtenheft, Spezifikation)
\item (Spezifiziert die Anforderungen an das Resultat der Arbeit)
\item (Übersicht über die Arbeit: stellt die folgenden Teile der Arbeit kurz vor)
\item (Angaben zum Zielpublikum: nennt das für die Arbeit vorausgesetzte Wissen)
\item (Terminologie: Definiert die in der Arbeit verwendeten Begriffe)
\end{itemize}
The aim of the project was to build and train a model which uses reinforcement learning to optimize train traffic on the flatland simulation.\\
%TODO: add verlinkung auf Aufgabenstellung im Anhang.
%TODO: PDF Aufgabenstellung (ZHAW und Flatland?)
The work it self consists out of 3 major parts: first round, second round and the attempt to add communication between the agents to prevent them from blocking each other. \\

