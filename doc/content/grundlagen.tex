%%%%%%%%%%%%%%%%%%%%%%%%%%%%%%%%%%%%%%%%%%%%%%%%%%%%%%%%%%%%%%%%%
%  _____   ____  _____                                          %
% |_   _| /  __||  __ \    Institute of Computitional Physics   %
%   | |  |  /   | |__) |   Zuercher Hochschule Winterthur       %
%   | |  | (    |  ___/    (University of Applied Sciences)     %
%  _| |_ |  \__ | |        8401 Winterthur, Switzerland         %
% |_____| \____||_|                                             %
%%%%%%%%%%%%%%%%%%%%%%%%%%%%%%%%%%%%%%%%%%%%%%%%%%%%%%%%%%%%%%%%%
%
% Project     : LaTeX doc Vorlage für Windows ProTeXt mit TexMakerX
% Title       : 
% File        : grundlagen.tex Rev. 00
% Date        : 7.5.12
% Author      : Remo Ritzmann
% Feedback bitte an Email: remo.ritzmann@pfunzle.ch
%
%%%%%%%%%%%%%%%%%%%%%%%%%%%%%%%%%%%%%%%%%%%%%%%%%%%%%%%%%%%%%%%%%

\chapter{Technical and mathematical foundation}\label{chap.grundlagen}
\section{Intro into reinforcement learning}\label{projektmanagement}
In recent years, major progress in reinforcement learning has been achieved.
In reinforcement learning, an agent $\mathcal{A}$ learns to perform a task by interacting with an environment $\mathcal{E}$.
If the agent does well, it receives positive reward from the environment, if it does something bad, there is no or negative reward. The goal of reinforcement learning algorithm is now to maximize the expected future reward $\mathbf{E}[\mathcal{R}_{t+1}+\mathcal{R}_{t+1}+\mathcal{R}_{t+1}+...|\mathcal{S}_{t}]$
Policy based reinforcement learning aims to aquire a policy $\pi$ that maximizes the received reward $\mathcal{R}$. This happens by selecting an action 



Mnih et al, DQN Atari
https://www.cs.toronto.edu/~vmnih/docs/dqn.pdf

Wu et al, A3C
https://arxiv.org/abs/1602.01783

Overview over MARL, Hernandez-Leal et al
https://arxiv.org/pdf/1810.05587.pdf

A3C in a multi agent environment, 
https://arxiv.org/pdf/1903.01365.pdf

