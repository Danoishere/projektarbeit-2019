%%%%%%%%%%%%%%%%%%%%%%%%%%%%%%%%%%%%%%%%%%%%%%%%%%%%%%%%%%%%%%%%%
%  _____   ____  _____                                          %
% |_   _| /  __||  __ \    Institute of Computitional Physics   %
%   | |  |  /   | |__) |   Zuercher Hochschule Winterthur       %
%   | |  | (    |  ___/    (University of Applied Sciences)     %
%  _| |_ |  \__ | |        8401 Winterthur, Switzerland         %
% |_____| \____||_|                                             %
%%%%%%%%%%%%%%%%%%%%%%%%%%%%%%%%%%%%%%%%%%%%%%%%%%%%%%%%%%%%%%%%%
%
% Project     : LaTeX doc Vorlage für Windows ProTeXt mit TexMakerX
% Title       : 
% File        : resultate.tex Rev. 00
% Date        : 23.4.12
% Author      : Remo Ritzmann
% Feedback bitte an Email: remo.ritzmann@pfunzle.ch
%
%%%%%%%%%%%%%%%%%%%%%%%%%%%%%%%%%%%%%%%%%%%%%%%%%%%%%%%%%%%%%%%%%

\chapter{Results}\label{chap.resultate}
\section{Round 1}
Our submission for Flatland round 1 does not include all algorithmic improvements discussed in this work. None the less, we were able to achieve a significant improvement in performance compared to the baseline version from \cite{flatlandstephan}.
Our submission for round 1 contains the following components:
\begin{itemize}
	\item Custom A3C implementation without experience replay buffer.
	\item Default TreeObsForRailEnv (in round 1, this was a numeric vector by default).
	\item Policy learned by curriculum learning.
	\item Distributed training over multiple processes on the same machine (no cross-machine distribution possible yet).
\end{itemize}
Using the performance evaluation system provided together with the Flatland environment, we reach the following performance metrics:




%Vergleich mit Stefan:
%He used the first version of Flatland to evaluate his models and got a total score of 24.7\% in a local evaluation.
%He trained his model with 2 to 10 agents with a field of view of 10*10.

\section{Round 2}


