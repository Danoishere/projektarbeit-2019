%%%%%%%%%%%%%%%%%%%%%%%%%%%%%%%%%%%%%%%%%%%%%%%%%%%%%%%%%%%%%%%%%
%  _____   ____  _____                                          %
% |_   _| /  __||  __ \    Institute of Computitional Physics   %
%   | |  |  /   | |__) |   Zuercher Hochschule Winterthur       %
%   | |  | (    |  ___/    (University of Applied Sciences)     %
%  _| |_ |  \__ | |        8401 Winterthur, Switzerland         %
% |_____| \____||_|                                             %
%%%%%%%%%%%%%%%%%%%%%%%%%%%%%%%%%%%%%%%%%%%%%%%%%%%%%%%%%%%%%%%%%
%
% Project     : LaTeX doc Vorlage für Windows ProTeXt mit TexMakerX
% Title       : 
% File        : vorgehen.tex Rev. 00
% Date        : 7.5.12
% Author      : Remo Ritzmann
% Feedback bitte an Email: remo.ritzmann@pfunzle.ch
%
%%%%%%%%%%%%%%%%%%%%%%%%%%%%%%%%%%%%%%%%%%%%%%%%%%%%%%%%%%%%%%%%%

\chapter{Approach and methodology}\label{chap.vorgehen}
\section{Basic considerations}\label{basic_cons}
As described under (ref to first mention), our work is based on the work of S. Huschauer (REF). We take his idea of using the A3C algorithm to solve the flatland problem and try various modifications in an attempt to improve its performance. We proceed by giving an intution, what we want to achieve by changing the specified part, followed by an experiment to either prove or disprove our hypothesis.\\
For training purposes, we start by reimplementing the algorithm by ourselfes. This enabled us from the beginning to gain a deeper understanding of how the algorithm works and where we could find potential areas for improvement.


\section{Enhanced observations}\label{enhanced_observations}
Für die vorliegende Arbeit wurden die unten aufgeführten Programme eingesetzt.

\section{Distrubuted architecture & parallelism}\label{dist_architecture}
Für die vorliegende Arbeit wurden die unten aufgeführten Programme eingesetzt.

\section{Action space reduction}\label{reduced_action_space}
Für die vorliegende Arbeit wurden die unten aufgeführten Programme eingesetzt.

\section{Agent communication}\label{reduced_action_space}
Für die vorliegende Arbeit wurden die unten aufgeführten Programme eingesetzt.





\begin{itemize}
\item (Beschreibt die Grundüberlegungen der realisierten Lösung (Konstruktion/Entwurf) und die Realisierung als Simulation, als Prototyp oder als Software-Komponente)
\item (Definiert Messgrössen, beschreibt Mess- oder Versuchsaufbau, beschreibt und dokumentiert Durchführung der Messungen/Versuche)
\item (Experimente)
\item (Lösungsweg)
\item (Modell)
\item (Tests und Validierung)
\item (Theoretische Herleitung der Lösung)
\end{itemize}



\section{(Verwendete Software)}\label{software}
Für die vorliegende Arbeit wurden die unten aufgeführten Programme eingesetzt.

\subsection*{Arbeitsumgebung}\label{wintool}
\begin{itemize}
	\item Microsoft Windows 8 developer preview
\end{itemize}

\subsection*{Virtual Machine}\label{vm}
\begin{itemize}
	\item Oracle VM VirtualBox, Version 3.2.10
\end{itemize}

\subsection*{CAD Catia}\label{catia}
\begin{itemize}
	\item CATIA, Version 5.19 (in VirtualBox)
\end{itemize}

\subsection*{Dokumentation}\label{dokutools}
\begin{itemize}
	\item proTeXt mit TexMakerX 2.1 (SVN 1774), \href{http://www.latex-project.org/ftp.html}{latex-project.org}
	\item Microsoft Visio 2007
	\item Adobe Acrobat 8 Professional 8.1.6
\end{itemize}
