%%%%%%%%%%%%%%%%%%%%%%%%%%%%%%%%%%%%%%%%%%%%%%%%%%%%%%%%%%%%%%%%%
%  _____   ____  _____                                          %
% |_   _| /  __||  __ \    Institute of Computitional Physics   %
%   | |  |  /   | |__) |   Zuercher Hochschule Winterthur       %
%   | |  | (    |  ___/    (University of Applied Sciences)     %
%  _| |_ |  \__ | |        8401 Winterthur, Switzerland         %
% |_____| \____||_|                                             %
%%%%%%%%%%%%%%%%%%%%%%%%%%%%%%%%%%%%%%%%%%%%%%%%%%%%%%%%%%%%%%%%%
%
% Project     : LaTeX doc Vorlage für Windows ProTeXt mit TexMakerX
% Title       : 
% File        : vorgehen.tex Rev. 00
% Date        : 7.5.12
% Author      : Remo Ritzmann
% Feedback bitte an Email: remo.ritzmann@pfunzle.ch
%
%%%%%%%%%%%%%%%%%%%%%%%%%%%%%%%%%%%%%%%%%%%%%%%%%%%%%%%%%%%%%%%%%

\chapter{Vorgehen / Methoden}\label{chap.vorgehen}


\begin{itemize}
\item (Beschreibt die Grundüberlegungen der realisierten Lösung (Konstruktion/Entwurf) und die Realisierung als Simulation, als Prototyp oder als Software-Komponente)
\item (Definiert Messgrössen, beschreibt Mess- oder Versuchsaufbau, beschreibt und dokumentiert Durchführung der Messungen/Versuche)
\item (Experimente)
\item (Lösungsweg)
\item (Modell)
\item (Tests und Validierung)
\item (Theoretische Herleitung der Lösung)
\end{itemize}

\section{(Used Software)}\label{software}
We used the following tools in our project.

\subsection*{Working Environment}\label{os}
\begin{itemize}
	\item Microsoft Windows 10
	\item Ubuntu 19.04
\end{itemize}

\subsection*{Visual Studio Code}\label{vsc}
\begin{itemize}
	\item Visual Studio Code 1.40
\end{itemize}

\subsection*{Documentation}\label{dokutools}
\begin{itemize}
	\item XeLateX with Visual Studio Code
	\item XeLateX with WebStorm
\end{itemize}


\subsection*{Programming language}\label{programminglanguages}
\begin{itemize}
	\item Python 3.6
\end{itemize}

\subsection*{Python modules}\label{modules}
\begin{itemize}
	\item Flatland-rl 1.3 - 2.1.10
	\item Tensorboard 2.0
	\item Keras x.x
	\item Cython x.x
	\item %TODO: finish this
\end{itemize}


\section{Basic considerations}
%split into round 1 / round 2

\subsection{Round 1}
%Observations
%Convolutional network + Global observations
%Early stopping
%Reward vergabe angepasst
We started with rebuilding the A3C algorithm from S. Huschauer to get a better knowledge how A3C works.\\
We made some experiments with different observations: TreeObservations and GlobalOberservations.
Because we made better and faster progress with GlobalObservatione we continued with those and combind them with a convolutional network.
Right after starting this project, we faced a problem regarding the reward distribution.\\


\subsection{Round 2}




\section{Measurands}
%Messgrössen: Evaluator, benchmark

\section{Experiments}
%

\section{Solution approach}
%Neue Actions


\section{Testing and submissions}

\section{Theoretical derivation of the solution}



